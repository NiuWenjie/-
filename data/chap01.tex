\chapter{绪论}

\section{前言}
海洋是地球生态系统不可或缺的组成部分~\cite{王晓红2003人类活动对海洋生物多样性的影响},对可持续发展至关重要。海洋为人类创造了可持续的生计和体面的工作,对消除贫困起到了重要的作用。超过30亿人依靠海洋和沿海资源谋生。此外,海洋对于全球粮食安全和人类健康至关重要。它们还是全球气候的主要调节者,并为我们提供水和呼吸的氧气。最后,海洋蕴藏着巨大的生物多样性资源。

为了使海洋和海洋资源成功地促进人类福祉,需要生态系统的完整性以及适当的生物、地球、化学和物理过程。生态系统完整性允许提供所谓的支持性生态系统服务,而反过来又是重要的调节、供应文化生态系统服务的基础,对人类至关重要。尽管海洋和海洋资源提供的惠益对所有人都很重要,但高度依赖自然资源和生态系统服务的穷人,土著人民和弱势群体的福祉可能与这些惠益紧密相关。海洋之间的联系,海洋和海洋资源与人类福祉并非单方面。虽然人类福祉的增加常常以生态系统的完整性为代价,但它也有可能减少人为对海洋环境的负面影响,例如,对资源的可持续利用~\cite{孙富行2000资源水利与水资源可持续发展},生产和消费方式的变化以及改进人类活动的管理和控制。

海洋和海洋资源日益受到人类活动的威胁、退化或破坏,从而降低了它们提供关键生态系统服务的能力。重要的威胁类别包括:气候变化、海洋污染、对海洋资源的不可持续开采以及自然改变和破坏海洋和沿海生境和景观。沿海和海洋生态系统和生境的恶化正在对全世界人类的福祉产生不利影响。

没有保护和养护地球生态系统就无法实现人类福祉。为了维持海洋提供给人类的生活质量,同时维持其生态系统的完整性,将需要改变人类如何看、管理和利用海洋和海洋资源。

\section{研究意义}
海洋蕴藏着丰富的资源,如生物资源、油气资源、固体矿产资源、海水资源、海洋能源等~\cite{张耀光2006中国海洋经济与可持续发展}。这些资源将为人力资源短缺提供巨大的物质支撑。但是,除了传统的海洋生物资源外,其他资源的开发利用基本上都处于起步阶段。因此,海洋资源不合理开发影响着人类可持续发展。合理开发海洋资源、大力发展海洋产业、促进海洋经济发展,是解决当下资源、环境压力最可靠、最有效的途径之一~\cite{2016xcc}。如何以经济有效的方式去开发、探索和保护海洋是主要的挑战之一。而其中的关键就在于如何使用成本效益高的海洋观测技术和部署策略去研究海洋生物、地球、化学和生物学。

迄今为止,成功的降低成本和提高数据分辨率策略的共同特点是:(1)小型化;(2)使用价格更低的材料;(3)多传感器集成的创新方法;以及(4)让公民科学家,利益相关者和公众参与部署。除了上面讨论的传感元件的各个方面之外,所有海洋传感器还存在一系列其他补充挑战,在这些挑战中,创新有可能超越进步。例如,与典型的可在表面展开的传感器相比,水下传感器~\cite{朱光文2000海洋环境监测与现代传感器技术,李彦2006海洋监测传感器网络概念与应用探讨,王骥2008基于无线传感器网络的海洋环境监测系统研究,刘妹琴2021基于水下传感器网络的目标跟踪技术研究现状与展望}的包装考虑因素对于承受高压腐蚀性环境(对于传感元件以及电池,控制器和相关外围设备)而言至关重要。但是,压力壳体和水下连接器仍有足够的标准化空间,以进一步节省成本。在海洋学部署中,必须考虑物理和生物结垢,并且尽管普遍使用各种基于铜的策略和物理清除结垢,但仍然迫切需要创新的无毒方法。

海洋学部署的另一个常见挑战是部署平台对传感器的尺寸和重量施加了限制。这可以通过传感器自身的小型化或通过进一步创新平台设计本身来缓解。平台设计的进步也可能允许更长的部署时间,因此迫切需要电池技术或传感器的能效方面的创新,因为电池通常代表海洋学部署的体积和重量的大部分。传感器的数据收集和传输也可能受到功耗的极大限制。

部署越来越多的传感器和平台的首要问题是长期功耗和数据传输。随着更多传感器的部署,此问题可能变得越来越重要。因此,随着具有成本效益的传感器的不断发展,必须考虑传感器节点的战略发展。随着处理器的功率需求相对于数据传输的功率需求持续更快地降低,本地数据处理将变得越来越有利。未来的传感器可能潜在地在本地处理信息,例如,在本地聚合数据,以馈送到目标模型的机载实例,从而通过仅中继关键模型输出而不是数据传输来节省能源和成本。

本设计的目标是开发具有成本效益的、创新、紧凑、集成的低功耗海洋传感器系统。可从移动式和固定式海洋观测平台进行部署。它着力于追求一体化和低功耗,具有轻松集成和便捷实用的特点。最终实现对海洋生态和水文环境的有效监测,以期对海洋环境探索、海洋资源合理利用、开发和海洋环境保护起到积极作用。整个系统主要分为微处理器部分和海洋传感器部分。控制单元以MSP430系列超低功耗处理器~\cite{davies2008msp430}为核心,采用集成式管理方式,海洋传感器单元搭载可以监测温度、盐度、深度、水下噪声信号等水文参数以及叶绿素、PH值、溶解氧等生态参数的海洋传感器,能够监测到海洋环境中水文、水质、动力和生态等参数变化,实现了对海洋水文和生态环境更加便捷的监测,对分析预警、提高各海域动态监测和综合管理能力有着关键作用。小型化的低功率传感器可以在最少的人工干预下进行长时间部署,尤其是在难以接近的环境中。通过实验室模拟环境进行数据采集、对比、分析等,选择多种降低功耗的方案,以达到低功耗、高性能的要求,对于海洋传感器发展和使用有着重要的意义。该设计方案的实现对推动海洋 科学发展也有重要意义,海洋科学研究中的重大发现和科学问题的解决往往都是在对某一个定点长期实时的观测上完成的~\cite{2017lqk},该设计正是为定点长期实时的海洋观测提供了一个有效的可行性方案。

\section{国内外发展现状}
\subsection{国外发展现状}
监测和研究海洋生物、地球、化学和生物学的一项基本挑战是,海洋信号通常具有较高的空间(毫米到数千公里)和时间(几秒到几十年)的可变性,而长期趋势可能隐藏在较大的短期自然环境中变异性。海洋辽阔,因此在经济上和物流上都面临挑战,需要部署仪器以同时覆盖正确位置的短期变化和长期趋势。在沿海海洋中,原位限制由于这些动态系统固有的较大异质性,生物地球化学和生物观测节点和资产可能会更加明显。

作为响应,已经开发并部署了区域到全球规模的原位海洋观测网络,这彻底改变了研究海洋生物、地球、化学和生物学的方式,并使人们对复杂的全球系统有了新的认识。这种网络的例子比比皆是,包括全球海洋观测系统(GOOS)~\cite{盖明举1997全球海洋观测系统介绍}、美国综合海洋观测系统(IOOS)~\cite{王春谊2012美国海洋观测系统分析}、海洋观测网(OOI)~\cite{李风华2019海底观测网的研究进展与发展趋势}、加拿大海底观测网(NEPTUNE)~\cite{王辉2019基于文献计量的加拿大海洋观测网发展态势分析}、日本先进实时海底区域监测网(ARENA)~\cite{漆随平2019海洋环境监测技术及仪器装备的发展现状与趋势}、生物地球化学-Argo(BGC-Argo)计划~\cite{claustre2020observing}、南部海洋碳和气候观测与模拟(SOCCOM)~\cite{johnson2017biogeochemical}以及海洋中遥感的EXPort过程(EXPORTS)~\cite{2020cjd}。但是,成本权衡通常要求这些网络各自设计为回答有限的问题集,例如长期与短期的可变性,空间或时间范围等。解决普遍存在的物理局限性的一种通用策略,是海洋观测所在的空间覆盖范围以及需要了解海洋的巨大变化优先考虑(a)有限数量的特定,以项目为目标的地点进行高时间分辨率研究(高时间覆盖率而低空间覆盖率)或(b)较大的空间覆盖范围限制了较短的时间分辨率。例如,OOI和OceanSITES8旨在描述独特或代表性地点的长期变化(例如,海洋温度变暖、海洋酸化和脱氧)和关键过程(例如,海洋界面之间边界处的交换和相互作用)的特征,而生物地球化学-Argo计划目的是在全球范围内以较低的时间分辨率来测量关键的生物、地球、化学参数(对于这些关键的地球化学参数已经存在强大的传感器技术)~\cite{bittig2019bgc}。

GOOS利用海洋观测框架来指导其实施一体化和持续的海洋观测系统。这种系统方法旨在灵活并适应不断发展的科学、技术和社会需求,有助于提供具有最大用户基础和社会影响的海洋观测系统。全球海洋观测系统(GOOS)成功地协调了由GOOS原则统一的持续观测协作系统。凭借在联合国系统内的独特地位,GOOS能够调拨教科文组织/IOC会员国的资源,围绕独立管理和独立供资的观测要素(卫星、浮标、科学家等)建立网络。 

美国拥有目前全球技术最成熟的“海洋观测网”(OOI),该网络历时10年、耗资3.86亿美元并于2016年正式启动运行~\cite{2016wzj}。整个系统有着各种仪器,从基础的盐度传感器到复杂的水下滑翔机应有尽有,用来观测海底到海表的海洋水文和海洋生态参数变化。通过水下机器人、光电复合缆以及特定的观测仪器,把巨量的海洋数据实时、连续地传输到岸上。OOI观测这种全新的方式,能用更高的效率来进行科学研究,以及与海洋进行互动。OOI的观测数据可以有力的推动海洋科学研究的发展进步,提升科学家对复杂海洋学科之间的关系的认识。除了科学研究之外,对于基础教学提供海洋数据,大大提高了教学效率。同时对于海洋渔业也有显著的助力,可大大提高捕捞效率。

加拿大海洋观测网(ONC)是由东北太平洋的NEPTUE Canada观测网和VENUS海底实验站于2013年合并组建而成~\cite{2013hpy}。VENUS只是一个近海岸观测网,而对于NEPTUNE来说则意义非凡,它是全球首个大区域尺度、多节点、多传感器的海洋观测网~\cite{2017zjj}。其中NEPTUNE主要采用海底光电复合缆来进行海洋观测的构建,主体是一个软硬件集成控制系统,具备数据采集与传输、能源供应与分配以及数据管理与分析等功能。整个系统网络共有六个主节点,周围布放的接驳盒用以给传感器供电,有序的分布在浅海区域至深海区域,并且每个节点配备了6个湿插拔接口,可供仪器设备成阵或拓展~\cite{2020cjd}。最终通过挂载的海洋仪器对不同深度海底的生态环境变化、生物群落以及板块运动等进行长期连续的监测,所采集的数据对板块构造研究、海洋气候和生态系统等领域有着深远意义。

日本海洋观测网的主要特点是起步早,从2003年提出的实时海底观测网(ARENA),到海底观测密集网络(DONET)一期、二期工程的结束,以及2015年建成的海沟海底地震海啸观测网(S-net),主要是用来监测海底水压变化、板块构造等,以实现对地震和海啸等自然灾害进行预警目的。日本观测网组网技术成熟,整体架构以回路线缆结构为主,配备仪器设备辐射节点~\cite{2020cjd}。各仪器采用水下离合接口设计,一旦发生故障也不会对整体网络的运行造成任何影响,可使用水下自主机器人对出错的仪器进行更换即可~\cite{2013hpy}。

生物地球化学-Argo计划(BGC-Argo)于2016年正式启动,目标是在全球范围内测量关键的生物地球化学海洋变量。这些观测结果支持全球海洋观测系统(GOOS)所界定的三个主题内的目标:(i)气候变化,(ii)海洋生态系统健康和(iii)业务服务。在此背景下,BGC-Argo专门解决了五个科学主题(海洋酸化,氮循环和最低限氧区,生物碳泵,海洋碳吸收和浮游植物群落)和两个海洋管理主题(生物资源和碳预算验证)。是进行生物地球化学过程探索的独特方法。为了实现这些目标,BGC-Argo旨在组织和支持1000个仿形浮标网络的开发和持续运营。每个浮子将配备传感器以测量六个BGC-Argo核心变量:辐照度,悬浮颗粒,叶绿素a(chla),氧气(\ch{O2}),硝酸盐(\ch{NO3})和pH。2018年,这六个变量的测量得到国际海洋学委员会(IOC)的批准,以国际公认的方式和符合海洋法的方式为监测全球海洋铺平了道路。

NSF资助的南部海洋碳与气候观测与建模项目(SOCCOM)致力于释放南部海洋的奥秘并确定其对气候的影响。SOCCOM位于普林斯顿大学,由普林斯顿环境研究所管理,它利用美国各地调查人员的实力以及参与国际观测和模拟工作的优势。通过部署由约200个自主浮标组成的机器人观测系统来扩展稀疏的南大洋生物地球化学观测,该系统将在整个南大洋的时间和水平空间上提供近乎连续的覆盖,并在水柱深处提供垂直覆盖。利用这些观测数据,分析和改进新一代高分辨率地球系统模型,既可以增进我们对南大洋当前工作的了解,也可以更好地预测地球气候和生物地球化学的未来轨迹。

从目前世界先进的海洋观测网和海洋传感器发展来看,先进的海洋传感设备和观测网的建设促进了海洋科学、电子工程等方面的融合,对海洋资源的开发利用、海洋生态系统的保护以及自然灾害的预报提供了第一手资料。

\subsection{国内发展现状}

从国外的海洋观测网建设角度来看,海洋观测的建设需要国家和地方的大力支持。虽然我国海洋观测技术起步晚,但是近年来,我国对于海洋的发展越来越重视,从“十一五”期间开始规划,“十二五”期间进行建设,“十三五”期间全力发展,已经为我国海洋观测技术的发展提供了重要的技术储备和经验积累。

“十一五”期间,由同济大学在科技部“863”计划的支持下于2009年正式建成东海小衢山观测试验站~\cite{2011xhp}。该试验站主要包括了传感器搭载平台、海底光电复合缆、主体框架、不同型号的水密接插头、海洋观测仪器以及集能源供应和通信传输于一体的接驳盒等。后期经过上海科技委资助,完成对水下硬件设备的升级工作,并对海洋数据采集以及传输等方面进行了改造,同时增添了一些新的海洋传感器来实现对海洋的多方面监测,最终于2011年升级为综合海底观测网。其中数据传输方式采用了多路CDMA输出技术和大容量无线数据输出技术,通过光纤网络向各个观测点提供能量并实现数据采集,为我国海底观测技术研究奠定了良好的基础~\cite{2020cjd}。

“十二五”期间,由中科院南海海洋所、声学所等多家单位联合研制的“南海海底观测实验示范网”在海南三亚海域建设完成,并于2013年5月正式投入运行。该系统在研制过程重点突破了高压直流输配电技术、网络信息传输技术、水声通信技术以及水下可拔插应用技术等~\cite{2019lfh}。同时通过搭载的海洋观测仪器,成功的监测到了海底动力环境、化学环境、生态环境等方面的数据状态,是我国海洋观测网史上首个真正意义和具备完整功能的海洋观测系统。

“十三五”期间,由中科院声学所牵头的“南海深海海底观测实验系统”于2016年9月建成,包括浙江大学、中天科技等12家单位参与,铺设长度约150km,深度达1800m,为后续海洋观测网的进一步发展积累了新的经验与技术。同时也为2017年正式立项的“海底科学观测网”项目奠定了基础,未来将进入全天候、全方位的海洋立体观测时代。同时,我国也先后出台了《海洋观测站点管理办法》、《海洋气象发展规划(2016年~2025年)》等多项海洋条例与规划,对加大我国海洋观测投入力度、提高海洋仪器设备先进性、规范海洋观测管理制度等起到了明显的推动作用。

经过近十多年的发展,我国的海洋观测系统已经初步具备全球海洋立体观测雏形,与发达国家海洋观测技术的差距也在慢慢减小。而在海洋观测系统中,其中的关键部分包括针对传感器集成系统的研发,怎样实现海洋仪器更方便快捷的集成搭载到海洋观测平台上,从而降低开发难度、提升性能,是未来海洋观测技术发展的一个方向。

\section{论文的主要研究内容和框架结构}
本文主要从低功耗海洋传感器集成系统的整体架构设计和软件程序设计两大方面进行了论述,围绕海洋环境数据采集为中心展开工作。首先对低功耗海洋传感器集成系统的整体架构做了简略介绍,对其中的功能以及性能进行了阐述;其次主要对系统硬件平台设计进行了详细讲解,重点论述了能源供应、数据采集及数据保存等;然后对系统的低功耗设计方法进行了分析;最后分模块介绍了系统软件平台的设计。

论文共分为七个章节:

第一章:前言。论述海洋观测的重要性,以海洋数据监测为中心提出低功耗传感器搭载平台的研究意义,阐述所采集的海洋水文数据对于海洋生态保护以及海洋资源合理开发利用有着重要意义。通过介绍国内外海洋观测系统的现状以及国家目前的海洋发展状况,提出了本文研究的重点。最后介绍了本文的章节安排。

第二章:低功耗海洋传感器集成系统的组成。本章节主要介绍了海洋传感器集成系统的设计目标和总体架构,系统组成包括传感器单元和微处理器单元。分别对系统硬件和软件平台框架进行了简要介绍。最后对海洋传感器的技术指标和相关命令进行了概括。

第三章:低功耗海洋传感器集成系统的低功耗设计方案。围绕了系统低功耗的设计目标,从嵌入式微处理器选型和电源管理两个方面,阐述了系统低功耗的设计思路。

第四章:低功耗海洋传感器集成系统相关硬件设计。介绍了传感器连接控制模块和微处理器模块,重点讲述了两个模块的电路设计,阐明了电路原理图的基本工作原理,同时说明了可靠性和冗余设计对于系统研究工作的重要性。

第五章:低功耗海洋传感器集成系统相关软件设计。在本章节中根据系统的四个功能模块进行了软件程序实现工作。介绍了在对外交互上,通过管理与配置模块和数据采集与通信模块,分别实现与管理人员、海洋传感器之间的信息监护。阐明了能源管理与控制模块的软件设计如何实现性能和低功耗指标,海洋传感器数据如何进行有效的保存和处理。

第六章:实验测试与数据分析。本章主要介绍了系统的实验室测试效果和实际的现场布放,并截取系统布放后的布放采集数据,以此分析总结。

第七章:论文总结与展望布放,对课题工作进行总结,并针对已有工作的不足进行分析并提出改进方法。
