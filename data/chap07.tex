\chapter{总结与展望}
\label{cha:chapter7}

\section{总结}
本文主要提供了一个海洋生态环境监测的实现方案,采用集成式管理方式,设计了低功耗海洋传感器集成系统的主体结构,重点讲述了系统平台的低功耗设计、硬件设计和软件设计,阐明了电源的有效控制、传感器连接控制以及数据采集中心和部分电路原理图,整体具有功耗低、结构清晰、开发周期短等优点。系统选取了合适的超低功耗微处理器MSP430系列来作为控制核心,并完成相关硬件平台搭建,通过软件平台控制来完成传感器的数据采集以及电源管理等任务,最终海洋水文数据的有效采集和本地保存等功能。系统整体运行稳定,通过海洋环境的实地布放,可以对海洋生态环境的变化实现长期稳定的监测,能够达到预期的效果。

本文主要完成了以下工作:

(1)首先介绍了完整的低功耗海洋传感器集成系统总体设计方案,并重点剖析了海洋传感器的整体架构以及微处理系统的设计框图。采用MSP430微处理器进行电源控制,通过定点启动电源来触发采集任务,当平台数据采集任务处于上一个工作周期与下一个工作周期的中间休眠期时,将关闭电源,使微微处理器与传感器等外部设备可以在间歇时刻休息,以此来进一步降低平台的功耗,保证了能源消耗的最小化。一般的传感器集成系统只是简单的设计多个接口,不同的传感器类型之间很难实现替换,本系统不同传感器之间替换,只需更换探头即可实现。设计了两种通信接口协议,既能保证一般使用RS232协议的传感器可进行连接,又能连接使用专门AML接口协议的传感器。在已有的硬件资源基础之上,通过串口拓展芯片可随时增加连接传感器个数,使得系统具备可拓展性。在可靠性方面数据本地保存采用了冗余设计,分别对两片存储单元进行读写,保证了平台数据的稳定性,最终提升系统的整体工作性能。

(2)MSP430系列超低功耗微处理器是低功耗海洋传感器集成系统的硬件核心,围绕着它讨论了多种的低功耗设计方案。首先在嵌入式微处理器选型上,综合比较了C8051系列、STM32系列和MSP430系列,在实现性能的条件下,MSP430是能源节省和高效控制的最优选择。其次从电源管理方法上,具体分析了系统电源管理,比如电源在不同模式下的开启和关闭,时钟工作频率在不同模式的切换。

(3)由于整个传感器集成系统只有一个任务处理中心,且传感器的数据采集过程以及状态管理的数据反馈等所有的工作任务重担将全部交给微处理器进行处理。在软件平台设计上,分模块进行程序设计,通过系统管理与配置模块和数据采集与通信模块,分别实现与管理人员、海洋传感器之间的信息交互。能源管理与控制模块的软件设计实现了电源的有效控制,数据存储和回收模块对海洋生态环境数据进行有效的保存和处理。

本文主要以集成式管理方式实现对海洋生态环境的数据采集任务,通过布放于海洋环境进行实验验证,能够完成相关工作任务,为海洋环境监测和保护领域提供了一套新的解决方案。
% 文献引用~\cite{he2016deep}

\section{展望}
%新的一天快快把论文写完! 回来亲亲我的宝贝!

本文中的低功耗传感器集成系统能够满足基本要求,可以实现海洋生态环境数据的采集、储存和回收等工作,并且能够在海域中长期稳定的运行,但由于时间关系和作者掌握知识的局限性,后期可以从以下方面对平台加以改进,使平台功能更加完善可靠。

\subsection{电源控制升级}
由于海边环境的复杂性以及天气等原因,为了进一步保护传感器集成系统的安全,可中加入电压采集模块,以此来监测系统最开始处的电源供应或转换部分的电压变化。在单片机的程序中设定电压阈值,当电压出现较大的波动时且在无人值守的情况下,将启动电源切断程序,直接从电源供应或转换处切断能源供给,进一步加强对系统的保护,避免因电压波动所带来的破坏。

\subsection{便捷化和小型化}
海洋观测系统往便捷化、小型化发展是必然趋势。可通过电路板的PCB布局、更小型的封装、多层板结合的方式缩小电路板的体积,同时在硬件软件设计上进一步降低系统的功耗,从而减少携带的电池量,大大减少舱体的体积和重量。