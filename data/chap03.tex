\chapter{系统低功耗设计}
低功耗海洋传感器集成系统的设计目标就是长期于海底环境中对海洋水文数据定时采样。由于系统仅靠电池组维持电源供应,必须考虑系统的低功耗设计方案。系统采样时,微处理器处于活跃工作状态,并且控制电源给海洋传感器设备和电子器件进行电源供应;系统不采样时,微处理器处于休眠状态并且控制电源关断对各个器件的电源供应;如需再次激活系统,只需预先将微处理器唤醒,然后再恢复对系统的电源供应。系统低功耗设计具体实现主要从嵌入式微处理器选型和电源管理系统这两方面来入手。

\section{嵌入式微处理器选型}
在深海观测系统中,因受环境等因素的影响,嵌入式微处理器的选型应该考虑的因素有低功耗性、稳定性、计算能力、片内资源、开发环境、成本等。目前,市场上针对不同的应用场景开发出了多种微处理器芯片,可根据不同应用场景、不同功能需求选择不同系列不同资源的微处理器芯片,迄今为止电子市场比较主流的嵌入式微处理器芯片有以下几种:

C8051系列,应用最广泛的8位单片机,最早由Intel推出,内部集成AD、串口、SPI 接口,其I/O口使用较为简单且具备JTAG仿真功能。~\cite{2018wxp}。

MSP430系列,是TI公司推出的一种16位超低功耗的混合信号处理器,特点是在以超低的功耗保持较为强大的处理能力,其架构与多种低功耗模式配置使用,是延长便携式应用电池寿命的很好选择。在超低功耗方面,其可灵活控制时钟的运行以降低功耗,时钟关闭时其电流最低可降低到0.1uA~\cite{2008wx}。

Cortex系列,是ARM公司针对不同的应用环境推出的Cortex系列微处理器芯片,其中Cortex-M3 是一种基于ARM V7架构的最新ARM32位处理器内核~\cite{zhao2017design}。Cortex系列微处理器一般都有很丰富的外设资源,同时还兼备实时性能和优异计算能力的特点。 Cortex-M3的电源管理方案通过NVIC支持Sleep Now、Sleepon Exit、SLEEPDEEP 三种睡眠模式,使其能够以较低的能耗实现超丰富的功能~\cite{wx} 。

该系统的微处理器芯片选型主要参考以上三个系列,其中应用广泛的C8051系列虽然满足低功耗、廉价的条件,但是其可扩展性较差,计算能力方面也有不足,且自我保护的能力很差,满足不了该系统的稳定性原则,因此不适合本系统。低功耗海洋传感器集成系统需要携带相对较少的电池能源以减少水密舱的体积增加存活率提高整个系统的生存周期,所以系统的微处理器芯片选型时,低功耗的优先级略大于计算能力和片内资源。因此系统选用了处理能力略低于Cortex-M系列但低功耗性能突出的MSP430系列中的MSP430F5438A微处理器。

\section{电源管理}
关闭系统的电源是最为节省能耗的措施。系统外部电路设备的电源可通过微处理器来控制关闭,就可以降低电池组的能耗。在很多MSP430的应用中,电子仪器都是在有限的时间内保持运行。在电子仪器非工作的时间内保持电源持续的供应,这一部分造成的能源浪费是完全可以避免。通过提高能源利用效率的手段来降低功耗。

在低功耗海洋传感器集成系统应用中,电源管理部分是整个系统节省能源、实现长期稳定工作的关键。MSP430F5438A微处理器可通过设置寄存器来控制CPU进入到不同的工作模式当中,休眠状态比活跃状态节省很大的能源消耗。同时可控制的DC/DC模块通过微处理器的IO口进行设置,IO口的输出信号连接DC/DC模块的Ctrl引脚,这样即可通过IO口输出高电平或者低电平来控制DC/DC模块的开启或关闭~\cite{2013xyd}。

整个系统的能源消耗由MSP430芯片和其他模块电路能源消耗构成。微控制器芯片的内部集成大量的模拟外设,这些外设的工作状态对于整个系统的能源消耗有很大的影响~\cite{2005xy}。内部的部件可分为模拟部件和数字器件,工作频率的变化对数字部件有很大的变化,而模拟部件的功耗在不同频率下几乎没有变化。模拟部件和数字部件的能源消耗之和相当于整个芯片的总功耗。
\begin{equation}
\begin{split}
\mathcal{P} = CV^2f
\end{split}
\label{eq:example}
\end{equation}   

本系统一直在关注低功耗的要求,但是低功耗到底如何来定性的衡量。由公式\ref{eq:example}可知,功耗由三个要素决定,分别是C(负载电容)、V(供电电压)、f(系统的工作频率)~\cite{2007hsj}。从中可以看出电源电压对功耗的影响是最为关键的,工作频率、负载电容的决定性则次之。而由于在系统设计当中,负载电容往往是不可控制的,不会在电路设计和器件选型中有意的去降低负载电容。在不影响整个系统性能的前提下,为了实现系统低功耗的要求,应该着力于研究如何降低供电的电压、时钟进行各种工作模式的切换以求降低时钟频率对功耗的影响。

\subsection{电源电压管理}
电源电压管理主要从两个方面来降低整个系统电源电压以实现降低整个系统功耗的目的。一方面设计采用的MSP430F5438A具有低电源电压范围(3.6V到低至1.8V);另外一方面,系统中可以对DC/DC模块进行程序控制,在给到相应DC/DC模块Ctrl引脚的信号为高电位时,即开启了此处的电源供应,这部分的负载才会消耗电源电压,否则就会一直处于关闭的状态,节省了能源消耗。

整个系统的电源消耗由微处理器芯片内部集成的很多模拟外设和其它模块电路能源消耗组成。因此芯片内外的外设工作情况对整个系统的能源消耗起到决定性的影响。 

\subsection{时钟切换控制}
系统工作频率方面,主要是通过对微处理器件进行合理的时钟切换来降低整个系统的功耗。在上一章的硬件平台设计中,详细的描述了MSP430的基本时钟系统,这里就不再赘述。作为一款超低功耗的微控制器,MSP430本身的多种工作模式就为应用者进行低功耗设计提供了极大的便利。MSP430系列微处理器具有如表~\ref{tab:CTC}所示的不同工作模式。

%\newcommand{\tabincell}[2]{\begin{tabular}{@{}#1@{}}#2\end{tabular}}  
%表格自动换行
\begin{table*}[ht]
\caption{工作模式}
\label{tab:CTC}
\centering
    \begin{tabular}{|c|c|c|c|c|c|}
        \toprule
        {\bf SCG1} & {\bf SCG0} & {\bf OSCOFF}& {\bf CPUFF} & {\bf 模式}&{\bf CPU和时钟状态}  \\      
%\bf表示字体加粗
        \hline
        0  & 0  & 0 & 0& 活动模式& \tabincell{c}{CPU、MCLK活动模式。ACLK活动。\\SMCLK选择性活动 (SMCLKOFF=0)。} \\
%需要分行的单元格的语句用\tabincell{c}{所填写第一行内容\\第二行内容···},可以根据需要换行,也不限定换多少行。
        \hline
      0  & 0  & 0 & 1& LPM0& \tabincell{c}{CPU、MCLK禁止。ACLK活动,\\SMCLK有选择的活动(SMCLKOFF=0)} \\
        \hline
      0  & 0  & 0 & 1& LPM1& \tabincell{c}{CPU、MCLK禁止。ACLK活动,\\SMCLK有选择的活动(SMCLKOFF=0)} \\
        \hline
      0  & 1  & 0 & 1& LPM2& \tabincell{c}{CPU、MCLK禁止。ACLK活动,\\SMCLK有选择的活动(SMCLKOFF=0)} \\
     \hline
      0  & 0  & 0 & 1& LPM3& \tabincell{c}{CPU、MCLK禁止。ACLK活动,\\SMCLK禁止} \\
     \hline
      0  & 0  & 0 & 1& LPM4& \tabincell{c}{CPU和所有的时钟都禁止} \\
        \bottomrule
    \end{tabular}
\end{table*}

激活模式(AM)是所有的时钟激活;待机模式(LPM3)下带有晶振的实时时钟(RTC),看门狗和电源监视器工作,完全RAM保持,快速唤醒;关闭模式(LPM4)下完全RAM保持、电源监视器工作、快速唤醒。不同工作模式下供电电流的差别是极大的~\cite{2005xy},不同工作模式下典型的电流值如表~\ref{tab:dl}所示。

\begin{table}[ht]
\caption{各种工作模式的典型供电电流值($\mu$A)}
\label{tab:dl}
\centering
    \begin{tabular}{|c|c|c|}
        \hline
        \diagbox{\bf$\mu$A}{\bf 电压} & 2.2V & 3.0V \\   
        %斜线命令语句
        \hline
        I(AM) &280 &420 \\
         \hline
        I(LPM0) &32 &55 \\
         \hline
        I(LPM2) &11 &17 \\
         \hline
        I(LPM3) &0.9 &1.6 \\
         \hline
        I(LPM4) &0.1 &0.1 \\
         \hline
    \end{tabular}
\end{table}

工作模式的选择主要考虑超低功耗、强大的数据处理能力和外围模块功耗最小这三个方面的要求。MSP430F5438A具有五种低功耗模式(LPM0-LPM4)~\cite{2005wdy}和一种活动模式,低功耗模式LPM0-LPM4可以通过设置状态寄存器来实现。针对不同需求的应用场合,设计中应合理选择工作模式和时钟频率。比如在ADC周期性采样、串行通信等应用场合之下,需要在工作时间内保持主时钟频率。本系统中一般处于LPM3模式下,之所以选择LPM3模式是考虑到LPM4模式下CPU及所有的时钟(ACK)都禁止,但是系统要求软时钟在低功耗的模式下仍然可以使用且可中断唤醒CPU。任何使能的中断事件都可以把MSP430从低功耗模式(LPM0到LPM4)中唤醒。程序控制单片机在指定的时刻通过定时器中断进入活动模式。

\section{本章小结}
本章主要阐述了低功耗海洋传感器集成系统的低功耗设计,首先概述了低功耗设计的总体方案,然后对器件选型方面如何考虑低功耗要求进行分析,最后着重于描述系统电源管理,分别从降低供电电压和时钟切换控制方面来降低系统的能源消耗。






