\thusetup{
  %******************************
  % 注意:
  %   1. 配置里面不要出现空行
  %   2. 不需要的配置信息可以删除
  %******************************
  %
  % 中国海洋大学研究生学位论文封面
  % 参考:中国海洋大学研究生学位论文书写格式20130307.doc
  % 为避免出现错误,下面保留[清华大学学位论文模板原有定义无需修改],
  % 请直接跳到后面[中国海洋大学学位论文模板部分请根据自己情况修改]。
  %
%%%%%%%%%%%%%%%%%%%%%%[清华大学学位论文模板原有定义无需修改]%%%%%%%%%%%%%%%%%%%%%%%
  %=====
  % 秘级
  %=====
  secretlevel={秘密},
  secretyear={10},
  %
  %=========
  % 中文信息
  %=========
  ctitle={清华大学学位论文 \LaTeX\ 模板\\使用示例文档 v\version},
  cdegree={工学硕士},
  cdepartment={计算机科学与技术系},
  cmajor={计算机科学与技术},
  cauthor={薛瑞尼},
  csupervisor={郑纬民教授},
  cassosupervisor={陈文光教授}, % 副指导老师
  ccosupervisor={某某某教授}, % 联合指导老师
  % 日期自动使用当前时间,若需指定按如下方式修改:
  % cdate={超新星纪元},
  %
  % 博士后专有部分
  cfirstdiscipline={计算机科学与技术},
  cseconddiscipline={系统结构},
  postdoctordate={2009年7月——2011年7月},
  id={编号}, % 可以留空: id={},
  udc={UDC}, % 可以留空
  catalognumber={分类号}, % 可以留空
  %
  %=========
  % 英文信息
  %=========
  etitle={An Introduction to \LaTeX{} Thesis Template of Tsinghua University v\version},
  % 这块比较复杂,需要分情况讨论:
  % 1. 学术型硕士
  %    edegree:必须为Master of Arts或Master of Science(注意大小写)
  %             “哲学、文学、历史学、法学、教育学、艺术学门类,公共管理学科
  %              填写Master of Arts,其它填写Master of Science”
  %    emajor:“获得一级学科授权的学科填写一级学科名称,其它填写二级学科名称”
  % 2. 专业型硕士
  %    edegree:“填写专业学位英文名称全称”
  %    emajor:“工程硕士填写工程领域,其它专业学位不填写此项”
  % 3. 学术型博士
  %    edegree:Doctor of Philosophy(注意大小写)
  %    emajor:“获得一级学科授权的学科填写一级学科名称,其它填写二级学科名称”
  % 4. 专业型博士
  %    edegree:“填写专业学位英文名称全称”
  %    emajor:不填写此项
  edegree={Doctor of Engineering},
  emajor={Computer Science and Technology},
  eauthor={Xue Ruini},
  esupervisor={Professor Zheng Weimin},
  eassosupervisor={Chen Wenguang},
  % 日期自动生成,若需指定按如下方式修改:
  % edate={December, 2005}
  %
  % 关键词用“英文逗号”分割
  ckeywords={\TeX, \LaTeX, CJK, 模板, 论文},
  ekeywords={\TeX, \LaTeX, CJK, template, thesis}
}

% 定义中英文摘要和关键字
\begin{cabstract}
  论文的摘要是对论文研究内容和成果的高度概括。摘要应对论文所研究的问题及其研究目
  的进行描述,对研究方法和过程进行简单介绍,对研究成果和所得结论进行概括。摘要应
  具有独立性和自明性,其内容应包含与论文全文同等量的主要信息。使读者即使不阅读全
  文,通过摘要就能了解论文的总体内容和主要成果。

  论文摘要的书写应力求精确、简明。切忌写成对论文书写内容进行提要的形式,尤其要避
  免“第 1 章……;第 2 章……;……”这种或类似的陈述方式。

  本文介绍清华大学论文模板 \thuthesis{} 的使用方法。本模板符合学校的本科、硕士、
  博士论文格式要求。

  本文的创新点主要有:
  \begin{itemize}
    \item 用例子来解释模板的使用方法;
    \item 用废话来填充无关紧要的部分;
    \item 一边学习摸索一边编写新代码。
  \end{itemize}

  关键词是为了文献标引工作、用以表示全文主要内容信息的单词或术语。关键词不超过 5
  个,每个关键词中间用分号分隔。(模板作者注:关键词分隔符不用考虑,模板会自动处
  理。英文关键词同理。)
\end{cabstract}

% 如果习惯关键字跟在摘要文字后面,可以用直接命令来设置,如下:
% \ckeywords{\TeX, \LaTeX, CJK, 模板, 论文}

\begin{eabstract}
   An abstract of a dissertation is a summary and extraction of research work
   and contributions. Included in an abstract should be description of research
   topic and research objective, brief introduction to methodology and research
   process, and summarization of conclusion and contributions of the
   research. An abstract should be characterized by independence and clarity and
   carry identical information with the dissertation. It should be such that the
   general idea and major contributions of the dissertation are conveyed without
   reading the dissertation.

   An abstract should be concise and to the point. It is a misunderstanding to
   make an abstract an outline of the dissertation and words ``the first
   chapter'', ``the second chapter'' and the like should be avoided in the
   abstract.

   Key words are terms used in a dissertation for indexing, reflecting core
   information of the dissertation. An abstract may contain a maximum of 5 key
   words, with semi-colons used in between to separate one another.
\end{eabstract}

% \ekeywords{\TeX, \LaTeX, CJK, template, thesis}
%%%%%%%%%%%%%%%%%%%%%%%%%%%%%%%%%%%%%%%%%%%%%%%%%%%%%%%%%%%%%%%%%%%%%%%%%%%%%%%%

%%%%%%%%%%%%%%%%%%[中国海洋大学学位论文模板部分请根据自己情况修改]%%%%%%%%%%%%%%%%%%%
% 中国海洋大学研究生学位论文封面
% 必须填写的内容包括(其他最好不要修改):
%   分类号、密级、UDC
%   论文中文题目、作者中文姓名
%   论文答辩时间
%   封面感谢语
%   论文英文题目
%   中文摘要、中文关键词
%   英文摘要、英文关键词
%
%%%%%[自定义]%%%%%
\newcommand{\fenleihao}{}%分类号
\newcommand{\miji}{}%密级 
                    % 绝密$\bigstar$20年 
                    % 机密$\bigstar$10年
                    % 秘密$\bigstar$5年
\newcommand{\UDC}{}%UDC
\newcommand{\oucctitle}{低功耗海洋传感器集成系统的设计与实现}%论文中文题目
\ctitle{低功耗海洋传感器集成系统的设计与实现}%必须修改因为页眉中用到
\cauthor{管芳松}%可以选择修改因为仅在 pdf 文档信息中用到
\cdegree{工学博士}%可以选择修改因为仅在 pdf 文档信息中用到
\ckeywords{\TeX, \LaTeX, CJK, 模板, 论文}%可以选择修改因为仅在 pdf 文档信息中用到
\newcommand{\ouccauthor}{管芳松}%作者中文姓名
%\newcommand{\ouccsupervisor}{姬光荣教授}%作者导师中文姓名
%\newcommand{\ouccdegree}{博\hspace{1em}士}%作者申请学位级别
%\newcommand{\ouccmajor}{海洋信息探测与处理}%作者专业名称
%\newcommand{\ouccdateday}{\CJKdigits{\the\year}年\CJKnumber{\the\month}月\CJKnumber{\the\day}日}
%\newcommand{\ouccdate}{\CJKdigits{\the\year}年\CJKnumber{\the\month}月}
\newcommand{\oucdatedefense}{                }%论文答辩时间
%\newcommand{\oucdatedegree}{2009年6月}%学位授予时间
\newcommand{\oucgratitude}{谨以此论文献给我的导师和亲人!}%封面感谢语
\newcommand{\oucetitle}{Design and Implementation of Low-Power Ocean Sensor Integrated System}%论文英文题目
%\newcommand{\ouceauthor}{Haiyong Zheng}%作者英文姓名
\newcommand{\oucthesis}{\textsc{OUCThesis}}
%%%%%默认自定义命令%%%%%
% 空下划线定义
\newcommand{\oucblankunderline}[1]{\rule[-2pt]{#1}{.7pt}}
\newcommand{\oucunderline}[2]{\underline{\hskip #1 #2 \hskip#1}}

% 论文封面第一页
%%不需要改动%%
\vspace*{5cm}
{\xiaoer\heiti\oucgratitude

\begin{flushright}
---\hspace*{-2mm}---\hspace*{-2mm}---\hspace*{-2mm}---\hspace*{-2mm}---\hspace*{-2mm}---\hspace*{-2mm}---\hspace*{-2mm}---\hspace*{-2mm}---\hspace*{-2mm}---~\ouccauthor
\end{flushright}
}

\newpage

% 论文封面第二页
%%不需要改动%%
\vspace*{1cm}
\begin{center}
  {\xiaoer\heiti\oucctitle}
\end{center}
\vspace{10.7cm}
{\normalsize\songti
\begin{flushright}
{\renewcommand{\arraystretch}{1.3}
  \begin{tabular}{r@{}l}
    学位论文答辩日期:~ & \oucunderline{1.8em}{\oucdatedefense} \\
    指导教师签字:~ & \oucblankunderline{5cm} \\
    答辩委员会成员签字:~ & \oucblankunderline{5cm} \\
    ~ & \oucblankunderline{5cm} \\
    ~ & \oucblankunderline{5cm} \\
    ~ & \oucblankunderline{5cm} \\
    ~ & \oucblankunderline{5cm} \\
    ~ & \oucblankunderline{5cm} \\
    ~ & \oucblankunderline{5cm} \\
  \end{tabular}
}
\end{flushright}
}

\newpage

% 论文封面第三页
%%不需要改动%%
\vspace*{1cm}
\begin{center}
  {\xiaosan\heiti 独\hspace{1em}创\hspace{1em}声\hspace{1em}明}
\end{center}
\par{\normalsize\songti\parindent2em
本人声明所呈交的学位论文是本人在导师指导下进行的研究工作及取得的研究成果。据我所知,除了文中特别加以标注和致谢的地方外,论文中不包含其他人已经发表或撰写过的研究成果,也不包含未获得~\oucblankunderline{7cm}(注:如没有其他需要特别声明的,本栏可空)或其他教育机构的学位或证书使用过的材料。与我一同工作的同志对本研究所做的任何贡献均已在论文中作了明确的说明并表示谢意。
}
\vskip1.5cm
\begin{flushright}{\normalsize\songti
  学位论文作者签名:\hskip2cm 签字日期:\hskip1cm 年 \hskip0.7cm 月\hskip0.7cm 日}
\end{flushright}
\vskip.5cm
{\setlength{\unitlength}{0.1\textwidth}
  \begin{picture}(10, 0.1)
    \multiput(0,0)(0.2, 0){50}{\rule{0.15\unitlength}{.5pt}}
  \end{picture}}
\vskip1cm
\begin{center}
  {\xiaosan\heiti 学位论文版权使用授权书}
\end{center}
\par{\normalsize\songti\parindent2em
本学位论文作者完全了解学校有关保留、使用学位论文的规定,并同意以下事项:
\begin{enumerate}
\item 学校有权保留并向国家有关部门或机构送交论文的复印件和磁盘,允许论文被查阅和借阅。
\item 学校可以将学位论文的全部或部分内容编入有关数据库进行检索,可以采用影印、缩印或扫描等复制手段保存、汇编学位论文。同时授权清华大学“中国学术期刊(光盘版)电子杂志社”用于出版和编入CNKI《中国知识资源总库》,授权中国科学技术信息研究所将本学位论文收录到《中国学位论文全文数据库》。
\end{enumerate}
(保密的学位论文在解密后适用本授权书)
}
\vskip1.5cm
{\parindent0pt\normalsize\songti
学位论文作者签名:\hskip4.2cm\relax%
导师签字:\relax\hspace*{1.2cm}\\
签字日期:\hskip1cm 年\hskip0.7cm 月\hskip0.7cm 日\relax\hfill%
签字日期:\hskip1cm 年\hskip0.7cm 月\hskip0.7cm 日\relax\hspace*{1.2cm}}

\newpage

\pagestyle{plain}
\clearpage\pagenumbering{roman}

% 中文摘要
%%[需要填写:中文摘要、中文关键词]%%
\begin{center}
  {\sanhao[1.5]\heiti\oucctitle\\\vskip7pt 摘\hspace{1em}要}
\end{center}
{\normalsize\songti

习近平总书记在党的十九大报告中明确要求“坚持陆海统筹,加快建设海洋强国”,为建设海洋强国再一次吹响了号角。本工作基于海洋科学研究和海洋环境安全保障的需求,对低功耗海洋传感器集成系统进行总体方案设计,通过集成不同的海洋传感器,可实现海洋生态环境数据高效率、多要素、多层次的定点长期观测和获取,为我国海洋环境观测提供一种经济、安全、高效的可行方案。 

本工作主要研究低功耗海洋传感器集成系统的相关技术,首先介绍了低功耗海洋传感器集成系统的整体框架。其次,以低功耗、高稳定性为基本设计原则,实现多种低功耗设计方案,然后针对系统的的具体功能技术指标进行电子电路设计搭建硬件观测平台,再以硬件为基础进行系统软件设计,介绍各个模块功能要求和实现流程。最后进行软硬件功能检测、实验室系统联调、海洋环境海试实验以验证低功耗海洋传感器集成系统工作的稳定性,实验结果满足系统性能指标,达到预期设计目标。

在硬件层面上,系统以MSP430F5438A为控制核心,该处理器具有超低功耗设计的特性,拥有丰富的接口资源,包括串口、SPI总线接口等,其中使用串口扩展芯片WK2124对板载资源进行了串口扩展,支持RS232或RS485通信方式,可以完成多串口数据收发功能,能够实现对海洋传感器的集成搭载,采集到的数据本地保存于冗余设计的两片TF卡当中,使用串口转USB接口芯片CH340E转化来的接口进行系统调试和数据回收,最终通过该系统可以控制海洋传感器对海洋环境的监测;传感器与微处理器之间采用集成式管理方式连接,以传感器连接控制模块为主,包括电源控制模块、数据通信模块等,主要是为海洋传感器进行供电,以及实现RS232或RS485电平与TTL之间的电平转换等。在软件层面上,分模块进行程序设计,通过系统管理与配置模块和数据采集与通信模块,分别实现与管理人员、海洋传感器之间的信息交互。能源管理与控制模块的软件设计实现了电源的有效控制,数据存储和回收模块对海洋生态环境数据进行有效的保存和处理。最终达到预期的设计目标,实现用经济、高效、安全的方式完成海洋生态环境数据的采集。

本工作的创新点是超低功耗的海洋传感器集成搭载方式以及多种海洋传感器之间任意替换的设计,可依据现场状况和需要采集的海洋生态环境要素选择传感器负载。在几秒钟内无需工具的情况下,即可在海上更改传感器的类型。多种通信协议电路以集成多类传感器设备并确保双向通信两端数据能安全可靠地无干扰传输。同时除了微处理器本身硬件接口外,使用串口拓展芯片进行了串口的拓展,满足了更多海洋传感器的集成,实现海洋水下温、盐、流、深、压剖面数据的定点、长期观测。同时整个系统数据格式统一化、指令规范化以及其接口的拓展性,为以后更加广泛的应用提供了可能。

}
\vskip12bp
{\xiaosi\heiti\noindent
关键词:超低功耗,海洋传感器,数据采集,多串口}

\newpage

% 英文摘要
%%[需要填写:英文摘要、英文关键词]%%
\begin{center}
  {\sanhao[1.5]\heiti\oucetitle\\\vskip7pt Abstract}
\end{center}
{\normalsize\songti

In the report of the 19th National Congress of the Communist Party of China, General Secretary Xi Jinping clearly demanded that “adhere to land and sea coordination and accelerate the construction of a maritime power”, which once again sounded the clarion call for building a maritime power. Based on the requirements of marine scientific research and marine environment security, this paper carries out the overall plan design, and designs and implements a low-power marine sensor integrated system. By integrating different marine sensors, it can achieve high-efficiency, multi-element, and multi-element marine ecological environmental data. Hierarchical fixed-point long-term observation and acquisition provide an economical, safe, and efficient feasible plan for my country's marine environment observation.

This article mainly studies the related technologies of low-power marine sensor integrated system, and first introduces the overall framework of low-power marine sensor integrated system. Secondly, based on the basic design principles of low power consumption and high stability, a variety of low power design schemes are realized, and then the electronic circuit design is based on the specific functional and technical indicators of the system to build a hardware observation platform, and then the system software is based on the hardware. Design, introduce the functional requirements and implementation process of each module. Finally, software and hardware function testing, laboratory system joint debugging, and marine environment sea test experiments are carried out to verify the stability of the low-power marine sensor integrated system. The experimental results meet the system performance indicators and achieve the expected design goals.

At the hardware level, the system uses MSP430F5438A as the control core. The processor has ultra-low power design features and has a wealth of interface resources, including serial ports and SPI bus interfaces. Among them, the serial port expansion chip WK2124 is used to perform onboard resources. Serial port expansion, support RS232 or RS485 communication mode, can complete the multi-serial data transmission and reception function, can realize the integrated loading of marine sensors, the collected data is stored locally in the redundant design of two TF cards, using the serial port to USB interface chip The interface converted by CH340E is used for system debugging and data recovery, and finally through the system, the monitoring of the marine environment by the marine sensor can be controlled; the sensor and the microprocessor are connected in an integrated management mode, and the sensor is connected to the control module mainly, including the power supply The control module, data communication module, etc., are mainly used to supply power for marine sensors, and realize the level conversion between RS232 or RS485 level and TTL. At the software level, the program is designed in modules, and the information interaction with managers and ocean sensors is realized through the system management and configuration module and the data acquisition and communication module. The software design of the energy management and control module realizes the effective control of the power supply, and the data storage and recovery module effectively saves and processes marine ecological environment data. In the end, the expected design goal is achieved, and the collection of marine ecological environment data can be completed in an economical, efficient and safe manner.

The innovation of this article is the ultra-low power consumption marine sensor integrated carrying method and the design of arbitrary replacement among multiple marine sensors. The sensor load can be selected according to the on-site conditions and the marine ecological environment elements that need to be collected. The type of sensor can be changed at sea in a few seconds without tools. Multiple communication protocol circuits to integrate multiple types of sensor equipment and ensure safe, reliable and interference-free transmission of data at both ends of the two-way communication. At the same time, in addition to the hardware interface of the microprocessor itself, the serial port expansion chip is used to expand the serial port, which satisfies the integration of more ocean sensors, and realizes the fixed-point and long-term observation of ocean underwater temperature, salt, current, depth, and pressure profile data. At the same time, the unification of the data format of the entire system, the standardization of instructions and the scalability of its interfaces provide the possibility for more extensive applications in the future.
}
 
\vskip12bp
{\xiaosi\heiti\noindent 
\textbf{
Keywords :  Ultra low power consumption,Ocean sensor,data collection,Multiple serial ports}}
%%%%%%%%%%%%%%%%%%%%%%%%%%%%%%%%%%%%%%%%%%%%%%%%%%%%%%%%%%%%%%%%%%%%%%%%%%%%%%%%
